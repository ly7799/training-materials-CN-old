\subchapter
{Autotools advanced}
{Objectives:
  \begin{itemize}
  \item Use {\tt AC\_ARG\_ENABLE} and {\tt config.h}
  \item Implement a shared library
  \item Switch to multiple directories
  \item Make the compilation of programs conditional
  \item Use {\tt pkg-config}
  \end{itemize}
}

\section{Use {\tt AC\_ARG\_ENABLE} and {\tt config.h}}

Continue to work on the project started in the previous lab. Start by
adding a \code{--enable-message} option in the \code{configure.ac} by
using \code{AC_ARG_ENABLE}. The purpose of this option, which should
be enabled by default, is to allow you to enable or disable the
printing of the {\em Hello World} message.

You will have to combine a {\em configuration header}, and
\code{AC_DEFINE} to let the C source code know whether the option was
enabled or not. \code{AC_DEFINE} could for example define a macro
named \code{WANTS_HELLO_WORLD}.

Once you are done, test that running \code{./configure} or
\code{./configure --enable-message} gives a \code{config.h} file with
\code{WANTS_HELLO_WORLD} defined, and that \code{./configure
  --disable-message} gives a \code{config.h} without
\code{WANTS_HELLO_WORLD}.

Change the \code{hello.c} program to use
\code{WANTS_HELLO_WORLD}. Build your \code{hello} program with
\code{--enable-message} and \code{--disable-message} consecutively,
and check that in the first case, \code{Hello World} is display, and
that in the second case \code{Hello World} is not displayed:

\begin{verbatim}
$ ./configure --enable-message
$ make
$ ./hello
Hello World
$ ./configure --disable-message
$ make
$ ./hello
$
\end{verbatim}

\section{Implement a shared library}

For the purpose of this training, we'll implement a small library
called \code{libhello}, which provides a single function
\code{show_msg()} responsible for displaying the \code{Hello World}
message. Our program will link against this library and use the
function it provides.

So create a file named \code{core.c}, with this function
\code{show_msg()} (it should continue to use the
\code{WANTS_HELLO_WORLD} macro defined in the previous section).

Create a header file named \code{hello.h}, which contains the
prototype of the \code{show_msg()} function.

Then, change the \code{hello.c} program so that it calls the
\code{show_msg()} function.

Once the source preparation is done, it's time to adapt the build
system:

\begin{enumerate}
\item Adapt the \code{configure.ac} script to initialize
  \code{libtool}
\item Change the \code{Makefile.am} to:
  \begin{itemize}
  \item build the \code{libhello} library from the \code{core.c} file
  \item declare the library version
  \item install the \code{hello.h} header
  \item make the \code{hello} program link against the \code{libhello}
    library
  \end{itemize}
\end{enumerate}

Once this is done, {\em autoreconf} your project, run
\code{configure}, build, and run the \code{hello} program.

Configure it with \code{--prefix=$HOME/sys}, and install it there. You
should see the library being installed in \code{$HOME/sys/lib} and the
header file in \code{$HOME/sys/include}. By running:

\begin{verbatim}
readelf -d $HOME/sys/bin/hello
\end{verbatim}

You can verify that \code{hello} is indeed linked against
\code{libhello}.

To finish up this part, if you look back at the \code{autoreconf -i}
output, it complained about \code{configure.ac} not using
\code{AC_CONFIG_MACRO_DIR}. So as explained in the slides, make the
needed changes to \code{configure.ac} and \code{Makefile.am}.

\section{Switch to multiple directories}

Even though our project is small, it's time to experiment with
subdirectories. Since we're modern, we'll use {\em non-recursive
  make}.

Move \code{core.c} and \code{hello.h} to a folder called \code{lib/},
and move \code{hello.c} to a folder called \code{src/}.

Adjust the main \code{Makefile.am} accordingly, and make sure
everything continues to build correctly.

Hint: you will have to add a \code{hello_CPPFLAGS} variable.

\section{Make the compilation of programs conditional}

Since some people may not be interested in building the \code{hello}
program, we'll make this optional. Create a new
\code{--enable-programs} option in \code{configure.ac}, which should
be enabled by default.

Then, use an {\em automake} conditional to build the \code{hello}
program only if enabled. Verify that when you pass
\code{--disable-programs}, the \code{hello} program is not built.

\section{Use {\tt pkg-config}}

Now we will make our program rely on the \code{libconfig} library. You
can install this library on your system, and its development files by
running:

\begin{verbatim}
apt install libconfig-dev
\end{verbatim}

Now, change \code{src/hello.c} to do the following:

\begin{verbatim}
#include <libconfig.h>
#include "hello.h"

int main(void)
{
   config_t cfg;
   config_init(&cfg);
   show_msg();
   return 0;
}
\end{verbatim}

It doesn't do anything useful, but calls one function of the
\code{libconfig} library, \code{config_init}, which is enough for our
demonstration.

If you try to build the program, it will fail with the following
error:

\begin{verbatim}
hello.c:7: undefined reference to `config_init'
\end{verbatim}

This is because we are not yet linking with the \code{libconfig}
library. We will use \code{pkg-config} to detect it and use the
appropriate linker flags. To do that, use \code{PKG_CHECK_MODULES} in
\code{configure.ac} when programs are enabled, and adjust your
\code{Makefile.am} to use the compiler and linker flags provided by
the \code{PKG_CHECK_MODULES} macro.

If everything works fine, you should see the \code{./configure} script
detecting \code{pkg-config}:

\begin{verbatim}
$ ./configure
[...]
checking pkg-config is at least version 0.9.0... yes
checking for LIBCONFIG... yes
[...]
\end{verbatim}

And finally the \code{hello} binary being linked with
\code{libconfig}:

\begin{verbatim}
$ make
[...]
libtool: link: gcc -g -O2 -o .libs/hello src/hello-hello.o  ./.libs/libhello.so -lconfig
[...]
\end{verbatim}

Congratulations, your program is now properly using the
\code{libconfig} library!

\section{Going further}

Implement the use of {\em silent rules} as explained in the slides,
and experiment with \code{make V=1} vs. \code{make V=0}.

Add a README file to your project, and make sure it is properly
distributed as part of the tarball generated by \code{make dist}

Use \code{AC_CONFIG_AUX_DIR} to store the {\em auxilliary files} in a
custom directory.
