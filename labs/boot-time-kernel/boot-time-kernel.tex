\subchapter{Kernel optimizations}{Measure kernel boot components and
optimize the kernel boot time}

\section{Measuring}

We are going to use the kernel \code{initcall_debug} functionality.

Our default kernel already has the configuration settings that we need:
\begin{itemize}
\item \kconfigval{CONFIG_PRINTK_TIME}{y}, to add a timestamp to each kernel
message.
\item \kconfigval{CONFIG_LOG_BUF_SHIFT}{16}, to have a big enough kernel ring buffer.
\end{itemize}

That's not sufficient. We also need the output of the \code{dmesg}
command.

We are going to make a few changes to the root filesystem. To save time
later going back to the initial Buildroot configuration, make a copy
of the \code{buildroot/} directory to \code{buildroot-dmesg/}:

\begin{verbatim}
rsync -aH buildroot/ buildroot-dmesg/
\end{verbatim}

In this new directory, add support for \code{dmesg} command in BusyBox,
and add the below line after the \code{ffmpeg} file in the
\code{playvideo} scripts:

\begin{verbatim}
dmesg > /dev/console
\end{verbatim}

Run Buildroot again, and update your
\code{~/boot-time-labs/rootfs/rootfs} directory again. Compile your
kernel again to to update the \code{zImage} with this root filesystem.

Now, let's enable \code{initcall_debug} in kernel parameters. Go to
the U-Boot command line, and add the below settings to the kernel command line
\footnote{Don't save these settings with \code{saveenv}. We
will just need them once.}, and boot your system:
\begin{verbatim}
setenv bootargs ${bootargs} initcall_debug printk.time=1
boot
\end{verbatim}

Boot the board with the new kernel image. If everything went well,
you can now copy and paste the special \code{dmesg} output to
a \code{~/boot-time-labs/kernel/initcall_debug.log} file on your workstation.

In \code{~/boot-time-labs/kernel} (at least where the kernel sources
are), run the following command to generate a boot graph:

\begin{verbatim}
linux/scripts/bootgraph.pl initcall_debug.log > boot.svg
\end{verbatim}

You can view the boot graph with the \code{inkscape} vector graphics
editor:

\begin{verbatim}
sudo apt install inkscape
inkscape boot.svg
\end{verbatim}

\begin{center}
\includegraphics[width=\textwidth]{labs/boot-time-kernel/boot.pdf}
\end{center}

Now review the longest initcalls in detail. Each label is the name of
a function in the kernel sources. Try to find out in which source file
each function is defined\footnote{You can do it with utilities such as
\code{cscope}, which your instructor will be happy to demonstrate,
or through our on-line service to explore the Linux kernel sources:
\url{https://elixir.bootlin.com}}, and what each driver corresponds
to.

Then, you can look the source code and:
\begin{itemize}
\item See whether you need the corresponding driver or feature at all.
If that's the case, just disable it.
\item Otherwise, try look for obvious causes which
would explain the very long execution time: delay loops (look for
\code{delay}, parameters which can reduce probe time but are not used,
etc).
\item There could also be features than could be postponed.
However, in our special case, we should
only need to keep kernel features that we need to run our video player.
However, in a real life system, the boot graph could indeed reveal
drivers which could be compiled as modules and loaded later.
\end{itemize}

Recompile and reboot the kernel, updating the boot graph until there is
nothing left that you can do.

When you are done exploiting data from the boot graphs, you can remove
\code{dmesg} support from BusyBox and remove this command too
from \code{playvideo}. Update your root filesystem and then kernel so
that we get back to the original situation. We no longer need
\code{initcall_debug}.

\section{Removing unnecessary functionality}

It's time to start simplifying the kernel by remove drivers and features
that you won't need.

Do this {\bf very progressively}. If you go too fast, you'll end up with a
kernel that doesn't boot any more, but you won't be able to tell which
parameter should have been kept.

Also, don't disable \kconfig{CONFIG_PRINTK} too early
as you would lose all the kernel messages in the console.

Also, for the moment, don't touch the options related to size and
compression, including compiling the kernel with {\em Thumb2}, as the
impact of each option could depend on the size of the kernel.

Make sure you go through all the possibilities covered in the slides, in
particular to enable \kconfig{CONFIG_EMBEDDED} to allow to unselect further
features that should be present on a general purpose
system\footnote{Here we have a very specific system and we don't have
to support programs that could be added in the future and could need
more kernel features}.

At the end, you can disable \kconfig{CONFIG_PRINTK}, and observe your
total savings in terms of kernel size and boot time.

Last but not least, try to find other ways of reducing the kernel size.
Go through the \code{.config} file and the kernel build log and look for
ideas to further reduce size and boot time.

\section{Optimizing required functionality}

The time has come to make final optimizations on our kernel, mainly
related to code size.

First, measure and write down your kernel size and the total boot time:

\begin{tabular}{| l | l | r |}
  \hline
  Kernel type & Kernel size & Total boot time \\
  \hline
  \hline
  ARM & & \\
  \hline
  Thumb2 & & \\
  \hline
\end{tabular}

Now, compile your kernel with \kconfig{CONFIG_ARM_THUMB}. Before you do
this, you could make a backup copy of your kernel source directory with
\code{cp -al}, as a full rebuild of the kernel will be needed, and we
may want to roll back later. Fortunately, thanks to our feature
reduction work, the full rebuild should be faster than in the earlier labs.

Write down the kernel size and total boot time in the above table,
and keep whatever option works best for you.

Then, continue by trying all the kernel compression schemes listed in
the below table:

\begin{tabular}{| l | l | r |}
  \hline
  Compression type & Kernel size & Total boot time \\
  \hline
  \hline
  Gzip & & \\
  \hline
  LMZA & & \\
  \hline
  XZ & & \\
  \hline
  LZO & & \\
  \hline
  LZ4 & & \\
  \hline
  None & & \\
  \hline
\end{tabular}

For the \code{None} row, there is no kernel configuration option, but
all you have to do is take the \code{arch/arm/boot/Image} file, rename
it to \code{zImage} on your SD card, and boot it. This option can make
sense when the CPU is very slow and the storage is quite fast (like when
you're booting Linux on a CPU emulated on an FPGA).

At the end, keep the option that gives you the best boot time, and
update the below table:

\begin{tabular}{| l | l | r |}
  \hline
  Step & Duration & Description \\
  \hline
  \hline
  U-Boot SPL & & Between \code{U-Boot SPL 2019.01} and \code{U-Boot 2019.01} \\
  \hline
  U-Boot & & Between \code{U-Boot 2019.01} and \code{Starting kernel} \\
  \hline
  Kernel + Init scripts & & Between \code{Starting kernel} and \code{Starting ffmpeg} \\
  \hline
  Application & & Between \code{Starting ffmpeg} and \code{First frame decoded} \\
  \hline
  \hline
  Total & & \\
  \hline
\end{tabular}

Note that we have merged the {\em Kernel} and {\em Init scripts} parts
(the latter being very short anyway), because the kernel is now silent.

At the end of this lab, you can remove the \code{buildroot-dmesg}
directory, which is no longer needed.
